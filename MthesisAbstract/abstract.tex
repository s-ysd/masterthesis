\documentclass[12pt,a4paper,twoside,openright,titlepage,draft]{report}
\usepackage[utf8]{inputenc}
\usepackage{amsmath}
\usepackage{amsfonts}
\usepackage{amssymb}
\usepackage{makeidx}
\usepackage[dvipdfmx]{graphicx}
\usepackage[dvipdfmx]{color}
\usepackage{layout}
\usepackage{lscape}
\usepackage{xfrac}
\usepackage{tabularx}
\usepackage{rotating}
\usepackage{layout}
\usepackage{longtable}
%\usepackage{todonotes}
\usepackage{booktabs}   % Showing the table from pandas
\usepackage{lscape}     % Enable rotating
\usepackage[colorlinks=true,
							citecolor=black,
							linkcolor=black,
							urlcolor=red,
							linktocpage=true,
							hyperfootnotes=false,
                            dvipdfmx]{hyperref}

\usepackage[square,sort,numbers,authoryear]{natbib}
%\usepackage{chapterbib}
\setlength\bibhang{.3in}


\usepackage{lineno}
\usepackage{setspace}
\onehalfspacing%
\usepackage{microtype}
\usepackage{color}
\usepackage{fancyhdr}
\usepackage[labelfont=bf]{caption}
% For electronic submission use:
\usepackage[inner=2.5truecm,outer=2.5truecm,top=1.3truecm=bottom=0truecm]{geometry}
% For printing use:
%\usepackage[inner=4cm,outer=2cm,top=2cm=bottom=2cm]{geometry}
\usepackage{tocbibind}


\renewcommand{\chaptermark}[1]{\markboth{\MakeUppercase{\thechapter. #1 }}{}}

\fancyhf{}
\fancyhead[RO]{\bfseries\rightmark}
\fancyhead[LE]{\bfseries\leftmark}
\fancyfoot[C]{\thepage}
\pagestyle{empty}

\newcommand\frontmatter{\pagenumbering{roman}}
\newcommand\mainmatter{\cleardoublepage\pagenumbering{arabic}}
\bibpunct{(}{)}{;}{a}{}{,}

\begin{document}
\begin{center}
{\Large \textbf{Estimation of galaxy star formation rate from \\ low radio frequency observations}}\\
\vspace{0.2cm}
{\large \textbf{Shuntaro YOSHIDA} (ID:\@261801444)}\\
\end{center}

Star formation rate (SFR) is one of the most fundamental quantities of a galaxy.
Recombination lines, far-ultraviolet, and infrared (IR) luminosities, for example, are commonly used observational indicators of the SFR (e.g., Kennicutt \& Evans 2012).
In addition, some references (e.g., Condon 1992) have suggested the use of low-frequency emission (at GHz or lower) as a possibly better-performing SFR indicator compared to the traditional indicators, since it is insensitive to dust extinction.
Such an indicator is desirable for deepening the understanding of galaxy evolution because the effect of dust might be the biggest source of uncertainty for the SFR estimation at high-$z$.
%Since the low-frequency emission is insensitive to the dust extinction and will be observed even from distant galaxies by the telescope with higher sensitivity and angular resolution, we anticipate its usefulness.
% An indicator that is unaffected by dust will be reliable for deepening the understanding of galaxy evolution.
However, we still do not understand the galaxy spectral energy distribution at low frequencies due to the lack of observational data up to now.
Low-frequency emission is often approximated as a power-law, and recent spatially resolved radio observations show that the spectral index shows a large variety depending on the location in a galaxy (Kapi\'{n}ska et al., 2017; For et al., 2018).
This suggests that the radio emission from star-forming galaxies may not have a simple frequency dependence.
Therefore, we must investigate how the relation between radio emission and star-formation (SF) activity varies across the low frequencies.

In this study, we investigate the frequency dependence of the global low-frequency radio and IR luminosity relation, as measures of SF activity in nearby star-forming galaxies.
We select galaxies from the Herschel Reference Survey (HRS) catalog (Boselli et al., 2010), which is assumed to be representative of local galaxies.
We identify 18 star-forming galaxies with high-quality radio data from the GaLactic Extragalactic All-sky MWA (GLEAM) survey catalog (Hurley-Walker et al., 2017).
The radio sources compiled by the catalog have 20 narrow bands at $72\mbox{--}231\,\mathrm{MHz}$, which allows for a more accurate examination of the frequency dependence.

Firstly, we find that a single power-law fitting is valid for modeling the relation of radio with IR luminosities from MWA frequencies to $1.5\,\mathrm{GHz}$.
The result suggests that the variation of the spectral index on the local spatial scales do not affect the global relation significantly.
Secondly, we find that SFR calculated from the low-frequency radio emission is consistent with SFRs obtained from IR luminosities.
We calculated the SFR in two ways: 1) use the fitting result of each galaxy as the calibration for the indicator, and 2) use the averaged quantities for calibration from all sampled galaxies.
%Another one is calculating SFR by applying the same calibration equation using the averaged parameters to all sample galaxies.
While the former method gives us more consistent results with a scatter of $\sim20\%$, the latter results in two times more scatter than the former.
Larger scatter in the latter method would be attributed to the intrinsic uncertainty of the calibration.
In conclusion, we propose to use the individual spectral energy distribution for calculating the radio SFR with less uncertainty.
This study provides a powerful tool for future radio surveys, like the Square Kilometre Array (SKA).

\end{document}
